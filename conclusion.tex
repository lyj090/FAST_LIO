\section{Conclusion}
\label{sec:conclusion}
\subsection{Main Contributions}
\begin{itemize}
    \item \textbf{Complete Reproduction of the FAST-LIO Core Workflow}: 
    
    Implemented a tightly-coupled state estimation algorithm based on Iterative Extended Kalman Filter (IEKF), including IMU forward propagation, LiDAR point cloud backward propagation for motion distortion compensation, residual construction, and iterative updates.
    \item \textbf{Validation of Computational Efficiency Optimization}: 
    
    Experimentally demonstrated that the new form of Kalman gain based on the Woodbury matrix identity \eqref{eq:new_kalman_gain} reduces computational complexity from $\mathcal{O}(m^3)$ to $\mathcal{O}(n^3)$, enabling real-time processing of over a thousand feature points on resource-constrained platforms (e.g., UAV onboard computers).
    \item \textbf{Robustness Validation Across Scenarios}: 
    
    Achieved sub-meter localization accuracy (RMSE = 0.210 m) in indoor (feature-degraded) and outdoor (high-dynamic) scenarios of the M3DGR dataset, demonstrating strong robustness to lighting changes, geometric degradation, and motion distortion.
\end{itemize}

\subsection{Algorithm Advantages}
Compared to traditional loosely-coupled or optimization-based methods, the FAST-LIO framework offers the following significant advantages:
\begin{itemize}
    \item \textbf{Enhanced Robustness through Tight Coupling}: Directly integrates raw LiDAR feature points with IMU states, avoiding the degradation issues of loosely-coupled methods caused by feature loss.
    \item \textbf{Precise Motion Distortion Compensation}: Utilizes a backward propagation mechanism to independently align time and pose for each LiDAR point, significantly improving point cloud geometric consistency.
    \item \textbf{Real-Time Efficiency}: Thanks to the dimensional optimization of the Kalman gain, the system can stably operate at a frequency of 10 Hz without blocking.
    \item \textbf{Minimal State Representation}: Defines error states on the Lie group manifold, using minimal parameterization (3 DoF) for attitude uncertainty, enhancing filter stability.
\end{itemize}

\subsection{Limitations and Future Work}
Nevertheless, several limitations remain. The method assumes high-frequency, low-noise IMU data; performance may degrade under severe vibration or with low-cost IMUs. Moreover, the absence of loop closure means long-term drift accumulates uncorrected, limiting scalability to large-scale mapping. The feature extraction module, while effective in structured environments, can still fail in textureless or highly repetitive scenes.

Looking ahead, future work could integrate FAST-LIO with a global optimization backend to enable loop closure and map refinement. Another promising direction is the fusion of visual or event-based sensors to enhance robustness in LiDAR-degenerate conditions. Additionally, deploying the system on embedded platforms with hardware-aware acceleration—such as fixed-point arithmetic or sparse matrix optimizations—would further broaden its applicability in real-world robotic systems. Ultimately, this project not only validates the theoretical elegance of FAST-LIO but also underscores the practical viability of tightly-coupled filtering for next-generation autonomous navigation.

\section{Project Management and Division of Work}
\label{sec:management}
This project was completed through the collaboration of three members, with specific responsibilities as follows:
\begin{itemize}
    \item Junda Wu: Responsible for overall project coordination and compilation of the theoretical section.
    \item Hongkang Cui: Responsible for experimental design and result analysis.
    \item Yongjian Liu: Responsible for writing the report documentation.
\end{itemize}
