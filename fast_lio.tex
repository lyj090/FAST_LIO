\documentclass[12pt]{article}

% =============== 中文支持(使用 xelatex 编译)===============
% \usepackage{ctex}  % 自动启用 UTF-8 和中文字体(Fandol / system fonts)
% 每段段首空两格
\usepackage{indentfirst}

% =============== 基础排版 ===============
\usepackage[a4paper, margin=2.5cm]{geometry}
\usepackage{amsmath, amssymb, amsthm}
\usepackage{graphicx}
\usepackage{caption, subcaption}
\usepackage{booktabs}
\usepackage{multirow}
\usepackage{float}

% =============== 算法排版 ===============
\usepackage{algorithm}
\usepackage{algpseudocode}

% =============== 超链接与参考文献 ===============
\usepackage{hyperref}
\hypersetup{
    colorlinks=true,
    linkcolor=blue,
    citecolor=red,
    urlcolor=cyan
}

\usepackage[backend=biber, style=numeric, sorting=none]{biblatex}
\addbibresource{references.bib}

\renewcommand{\abstractname}{Abstract}
\renewcommand{\contentsname}{Contents}
\renewcommand{\figurename}{Figure}
\renewcommand{\tablename}{Table}

% =============== 标题与作者 ===============
\title{FAST-LIO: Theory and Reproduction \\ \large Course Project Report of {\it Optimal Estimation}}
\author{Junda Wu \and Hongkang Cui \and Yongjian Liu}
\date{\today}

\begin{document}

\maketitle
\tableofcontents
\newpage
\section{Introduction}
\label{sec:introduction}
\subsection{Background}
Simultaneous Localization and Mapping (SLAM) is a fundamental capability for autonomous mobile robots, including unmanned aerial vehicles (UAVs). While visual (inertial) odometry (VO/VIO) offers a lightweight and low-cost solution with rich RGB information, it suffers from several limitations: it lacks direct depth measurements, demands significant computational resources for 3D reconstruction, and is highly sensitive to lighting variations.

LiDAR-based odometry, on the other hand, overcomes these challenges by providing accurate, direct depth measurements that are robust to illumination changes and enable efficient 3D geometric inference. Recent works such as FAST-LIO \cite{fast_lio} and demonstrate that tightly-coupled LiDAR-inertial odometry based on iterated Kalman filtering can achieve fast, robust, and high-precision state estimation, making LiDAR a compelling sensor modality for reliable SLAM in dynamic and real-world environments.

\subsection{Problem Motivation}
Despite the robustness of LiDAR sensing, practical LiDAR-inertial odometry systems encounter three critical challenges that degrade estimation performance:

\begin{enumerate}
    \item \textbf{Feature Degradation in Cluttered Environments}: In unstructured or texture-poor scenes (e.g., long corridors, open fields), LiDAR point clouds lack distinctive geometric features. Thus, the LiDAR-based solution easily degenerates.

    \item \textbf{Motion Distortion in Scanning Process}: A LiDAR scan is not instantaneous; points are sequentially sampled over tens to hundreds of milliseconds. If the platform is in motion during this interval, the resulting point cloud becomes geometrically distorted, which significantly corrupts feature extraction and scan alignment, especially in high-dynamic scenarios.

    \item \textbf{Computational Bottleneck in Tight Fusion}: Tightly-coupled frameworks fuse raw LiDAR feature points (often $m \gg 100$) directly with IMU states. The conventional Kalman gain computation requires inverting a $3m \times 3m$ measurement covariance matrix, an operation with $\mathcal{O}(m^3)$ complexity that quickly becomes infeasible for real-time operation on resource-constrained platforms.
\end{enumerate}

Together, these issues demand an estimation framework that is \textit{robust to feature scarcity}, \textit{immune to motion-induced geometric distortion}, and \textit{computationally efficient} even under dense feature integration.

\subsection{Project Objectives}
To address the above challenges, this project reproduces and analyzes the core estimation pipeline of FAST-LIO, with three tightly aligned technical objectives:

\begin{itemize}
    \item \textbf{Robustness via Tight Coupling}: Replace scan-registration-based loosely-coupled fusion with a \textit{tightly-coupled iterated Kalman filter} that directly fuses raw LiDAR feature points with IMU measurements. This bypasses the need for stable high-level features, thereby mitigating \textit{feature degradation} in cluttered environments.

    \item \textbf{Motion Distortion Compensation}: Implement a formal \textit{backward propagation} mechanism that, using IMU preintegration from the scan-end time, reconstructs the relative pose at each LiDAR point’s timestamp. This enables precise motion undistortion of the entire scan, resolving the \textit{motion distortion} problem at the measurement level.

    \item \textbf{Efficient Kalman Update}: Based on the \textit{Woodbury matrix identity}, the reformulated Kalman gain can be expressed as
    \[
    \mathbf{K} = (\mathbf{H}^T \mathbf{R}^{-1} \mathbf{H} + \mathbf{P}^{-1})^{-1} \mathbf{H}^T \mathbf{R}^{-1},
    \]
    which reduces matrix inversion from $\mathcal{O}(m^3)$ to $\mathcal{O}(n^3)$ (with $n = \dim(\text{state}) \ll m$), directly tackling the \textit{computational bottleneck}.
\end{itemize}

We evaluate the implemented algorithm through ROS-based simulations and experiments on the public M3DGR dataset~\cite{m3dgr}, assessing performance in terms of estimation accuracy, computational latency, and robustness, and validating these results against theoretical analysis.

\subsection{Organization}
\begin{figure}[H]
    \centering
    \includegraphics[width=1\linewidth]{media/workflow.pdf}
    \caption{Overview of the FAST-LIO estimation framework with IMU-LiDAR tight coupling, motion undistortion, and efficient Kalman update.}
    \label{fig:workflow}
\end{figure}

According to the workflow of the FAST-LIO framework (Fig. \ref{fig:workflow}), the remainder of this report is organized as follows: 
Section~\ref{sec:modeling} formally defines the state-space model of the IMU-LiDAR system; 
Section~\ref{sec:iekf} details the IEKF algorithm, including forward/backward propagation, residual computation, and iterative update; 
Section~\ref{sec:implementation} describes the simulation setup and implementation; 
Section~\ref{sec:results} presents quantitative results and performance analysis; 
and Section~\ref{sec:conclusion} concludes with contributions, limitations, and future work.

\section{Problem Definition and System Modeling}
\label{sec:modeling}

This section provides a formal description of the LiDAR-inertial system, including the state vector, inputs, observations, noise characteristics, and the discrete-time state-space model used for estimation.

\subsection{System Description}
\label{subsec:system_description}

The task is real-time 6-DoF pose estimation of a mobile robot (e.g., UAV or ground vehicle) using a tightly coupled LiDAR-IMU sensor suite. The system components are defined as follows:

\begin{itemize}
    \item \textbf{State vector $\mathbf{x}_k$}: The full state of the IMU in the global frame $G$ includes:
    \begin{itemize}
        \item Position $\mathbf{p}^G_I \in \mathbb{R}^3$ and orientation $\mathbf{R}^G_I \in SO(3)$;
        \item Velocity $\mathbf{v}^G_I \in \mathbb{R}^3$;
        \item IMU bias terms: gyroscope bias $\mathbf{b}_\omega \in \mathbb{R}^3$, accelerometer bias $\mathbf{b}_a \in \mathbb{R}^3$;
        \item Gravity vector $\mathbf{g}^G \in \mathbb{R}^3$ (estimated online for robustness in non-level environments).
    \end{itemize}
    The total state dimension is 18:
    \begin{equation}
        \mathbf{x} = \left[ \mathbf{R}^G_I, \mathbf{p}^G_I, \mathbf{v}^G_I, \mathbf{b}_\omega, \mathbf{b}_a, \mathbf{g}^G \right]^\top.
    \end{equation}

    \item \textbf{Input $\mathbf{u}_k$}: Raw IMU measurements at time $k$:
    \begin{itemize}
        \item Angular velocity: $\boldsymbol{\omega}^m_I = \boldsymbol{\omega}_I + \mathbf{b}_\omega + \mathbf{n}_\omega$;
        \item Specific force: $\mathbf{a}^m_I = \mathbf{a}_I + \mathbf{b}_a + \mathbf{n}_a$,
    \end{itemize}
    where $\mathbf{n}_\omega$ and $\mathbf{n}_a$ are zero-mean Gaussian white noises.

    \item \textbf{Observation $\mathbf{z}_k$}: A set of geometric feature points (planes or edges) extracted from a LiDAR scan. Each point $j$ is initially expressed in its local scan frame $L_j$.

    \item \textbf{Sources of uncertainty}:
    \begin{itemize}
        \item \textit{Process noise}: IMU measurement noise ($\mathbf{n}_\omega, \mathbf{n}_a$) and random-walk bias drift ($\mathbf{n}_{b\omega}, \mathbf{n}_{ba}$);
        \item \textit{Measurement noise}: Geometric noise $\mathbf{n}_{f_j}$ in LiDAR point positions;
        \item \textit{Modeling errors}: Unmodeled dynamics, calibration inaccuracies, and feature extraction errors.
    \end{itemize}
\end{itemize}

\subsection{State-Space Model}
\label{subsec:state_space_model}

The system is modeled as a continuous-discrete hybrid dynamical system.

Let $\mathcal{M}$ be the manifold of dimension $n$ in consideration (e.g., $\mathcal{M} = SO(3)$). Since manifolds are locally homeomorphic to $\mathbb{R}^n$, we can establish a bijective mapping from a local neighborhood on $\mathcal{M}$ to its tangent space $\mathbb{R}^n$ via two encapsulation operators $\boxplus$ and $\boxminus$\cite{calculate_def}:

\begin{align*}
    \boxplus &: \mathcal{M} \times \mathbb{R}^n \to \mathcal{M}; &
    \boxminus : \mathcal{M} \times \mathcal{M} \to \mathbb{R}^n \\
    \mathcal{M} = SO(3): & \quad \mathbf{R} \boxplus \mathbf{r} = \mathbf{R} \mathrm{Exp}(\mathbf{r}); &
    \mathbf{R}_1 \boxminus \mathbf{R}_2 = \mathrm{Log}(\mathbf{R}_2^\top \mathbf{R}_1) \\
    \mathcal{M} = \mathbb{R}^n: & \quad \mathbf{a} \boxplus \mathbf{b} = \mathbf{a} + \mathbf{b}; &
    \mathbf{a} \boxminus \mathbf{b} = \mathbf{a} - \mathbf{b}
\end{align*}

where $\mathrm{Exp}(\mathbf{r}) = \mathbf{I} + \frac{\|\mathbf{r}\|}{\|\mathbf{r}\|} \sin(\|\mathbf{r}\|) + \frac{\mathbf{r}\mathbf{r}^\top}{\|\mathbf{r}\|^2} (1 - \cos(\|\mathbf{r}\|))$ is the exponential map\cite{calculate_def}, and $\mathrm{Log}(\cdot)$ is its inverse map. For a compound manifold $\mathcal{M} = SO(3) \times \mathbb{R}^n$ we have:

\begin{equation*}
    \begin{bmatrix} \mathbf{R} \\ \mathbf{a} \end{bmatrix} \boxplus \begin{bmatrix} \mathbf{r} \\ \mathbf{b} \end{bmatrix} = \begin{bmatrix} \mathbf{R} \boxplus \mathbf{r} \\ \mathbf{a} + \mathbf{b} \end{bmatrix}; \quad
    \begin{bmatrix} \mathbf{R}_1 \\ \mathbf{a} \end{bmatrix} \boxminus \begin{bmatrix} \mathbf{R}_2 \\ \mathbf{b} \end{bmatrix} = \begin{bmatrix} \mathbf{R}_1 \boxminus \mathbf{R}_2 \\ \mathbf{a} - \mathbf{b} \end{bmatrix}.
\end{equation*}

From the above definition, it is easy to verify that:
\begin{equation*}
    (\mathbf{x} \boxplus \mathbf{u}) \boxminus \mathbf{x} = \mathbf{u}; \quad \mathbf{x} \boxplus (\mathbf{y} \boxminus \mathbf{x}) = \mathbf{y}; \quad \forall \mathbf{x}, \mathbf{y} \in \mathcal{M}, \; \forall \mathbf{u} \in \mathbb{R}^n.
\end{equation*}

\subsubsection{Continuous-Time Dynamics}

Assuming an IMU is rigidly attached to the LiDAR with a known extrinsic ${}^{I}\mathbf{T}_{L} = ({}^{I}\mathbf{R}_{L}, {}^{I}\mathbf{p}_{L})$. Taking the IMU frame (denoted as $I$) as the body frame of reference leads to a kinematic model:

\begin{equation}
    \begin{split}
        \dot{\mathbf{p}}^G_I &= \mathbf{v}^G_I, \\
        \dot{\mathbf{v}}^G_I &= \mathbf{R}^G_I (\mathbf{a}^m_I - \mathbf{b}_a - \mathbf{n}_a) + \mathbf{g}^G, \\
        \dot{\mathbf{R}}^G_I &= \mathbf{R}^G_I (\boldsymbol{\omega}^m_I - \mathbf{b}_\omega - \mathbf{n}_\omega)^\wedge, \\
        \dot{\mathbf{b}}_\omega &= \mathbf{n}_{b\omega}, \quad
        \dot{\mathbf{b}}_a = \mathbf{n}_{ba}, \quad
        \dot{\mathbf{g}}^G = \mathbf{0}.
    \end{split}
    \label{eq:continuous_dynamics}
\end{equation}

where ${}^G\mathbf{p}_I$, ${}^G\mathbf{R}_I$ are the position and attitude of IMU in the global frame (i.e., the first IMU frame, denoted as $G$), ${}^G\mathbf{g}$ is the unknown gravity vector in the global frame, $\mathbf{a}_m$ and $\boldsymbol{\omega}_m$ are IMU measurements, $\mathbf{n}_a$ and $\mathbf{n}_\omega$ are the white noise of IMU measurements, $\mathbf{b}_a$ and $\mathbf{b}_\omega$ are the IMU bias modelled as the random walk process with Gaussian noises $\mathbf{n}_{b_a}$ and $\mathbf{n}_{b_\omega}$, and the notation $[\mathbf{a}]_\wedge$ denotes the skew-symmetric matrix of vector $\mathbf{a} \in \mathbb{R}^3$ that maps the cross product operation.

\subsubsection{Discrete Model}
Based on the $\boxplus$ operation defined above, we can discretize the continuous model in \eqref{eq:continuous_dynamics} at the IMU sampling period $\Delta t$ using a zero-order holder. The resultant discrete model is:
\begin{equation}
    \mathbf{x}_{i+1} = \mathbf{x}_i \boxplus (\Delta t \mathbf{f}(\mathbf{x}_i, \mathbf{u}_i, \mathbf{w}_i)),
    \label{eq:discrete_state}
\end{equation}
where $i$ is the index of IMU measurements, the function $\mathbf{f}$, state $\mathbf{x}$, input $\mathbf{u}$ and noise $\mathbf{w}$ are defined as below:

\begin{equation}
    \begin{split}
    \mathcal{M} &= SO(3) \times \mathbb{R}^{15}, \quad \dim(\mathcal{M}) = 18 \\
    \mathbf{x} &\doteq \begin{bmatrix} {}^G\mathbf{R}_I^T & {}^G\mathbf{p}_I^T & {}^G\mathbf{v}_I^T & \mathbf{b}_\omega^T & \mathbf{b}_a^T & {}^G\mathbf{g}^T \end{bmatrix}^T \in \mathcal{M} \\
    \mathbf{u} &\doteq \begin{bmatrix} \boldsymbol{\omega}_m^T & \mathbf{a}_m^T \end{bmatrix}^T, \quad
    \mathbf{w} \doteq \begin{bmatrix} \mathbf{n}_{b\omega}^T & \mathbf{n}_{ba}^T & \mathbf{n}_\omega^T & \mathbf{n}_a^T \end{bmatrix}^T \\
    \mathbf{f}(\mathbf{x}_i, \mathbf{u}_i, \mathbf{w}_i) &= 
    \begin{bmatrix}
        \boldsymbol{\omega}_{m_i} - \mathbf{b}_{\omega_i} - \mathbf{n}_{\omega_i} \\
        {}^G\mathbf{R}_{I_i} (\mathbf{a}_{m_i} - \mathbf{b}_{a_i} - \mathbf{n}_{a_i}) + {}^G\mathbf{g}_i \\
        \mathbf{n}_{b\omega_i} \\
        \mathbf{n}_{ba_i} \\
        \mathbf{0}_{3\times1}
    \end{bmatrix}.
    \label{eq:discrete_dynamics} 
    \end{split}
\end{equation}

\subsection{Measurement Model}

For the $j$-th LiDAR feature point, the residual observation is:
\begin{equation}
    \mathbf{z}_j = h_j(\mathbf{x}_k) + \mathbf{v}_j = \mathbf{G}_j \left( {}^G\hat{\mathbf{p}}_{f_j} - {}^G\mathbf{q}_j \right) + \mathbf{v}_j,
    \label{eq:meas_model}
\end{equation}
where:
\begin{itemize}
    \item ${}^G\hat{\mathbf{p}}_{f_j}$ is the motion-compensated LiDAR point in the global frame (computed via backward propagation);
    \item ${}^G\mathbf{q}_j$ is a reference point on the nearest plane or edge in the map;
    \item $\mathbf{G}_j = 
    \begin{cases}
        \mathbf{u}_j^\top & \text{(plane feature)} \\
        [\mathbf{u}_j]_\times & \text{(edge feature)}
    \end{cases}$, with $\mathbf{u}_j$ the normal or direction vector;
    \item $\mathbf{v}_j \sim \mathcal{N}(\mathbf{0}, \mathbf{R}_j)$ is the measurement noise.
\end{itemize}

\subsection{Noise and Uncertainty Settings}
\label{subsec:noise}

The noise covariances are configured as:

\begin{itemize}
    \item \textbf{Process noise covariance $\mathbf{Q} \in \mathbb{R}^{15 \times 15}$}: Diagonal matrix:
    \[
    \mathbf{Q} = \mathrm{diag}\left(
        \sigma_{\omega}^2 \mathbf{I}_3,\,
        \sigma_{a}^2 \mathbf{I}_3,\,
        \sigma_{b\omega}^2 \mathbf{I}_3,\,
        \sigma_{ba}^2 \mathbf{I}_3
    \right),
    \]
    where $\sigma_{\omega}, \sigma_{a}, \sigma_{b\omega}, \sigma_{ba}$ are standard deviations from sensor specs.

    \item \textbf{Measurement noise covariance $\mathbf{R} \in \mathbb{R}^{m \times m}$}: Diagonal (feature points assumed independent):
    \[
    \mathbf{R} = \mathrm{diag}\left( \sigma_{\text{plane}}^2, \dots, \sigma_{\text{edge}}^2, \dots \right).
    \]
\end{itemize}

\subsection{Observability and Identifiability Analysis}
\label{subsec:observability}

The system is \textit{locally observable} under general motion:

\begin{itemize}
    \item IMU provides continuous excitation of position, velocity, and orientation;
    \item LiDAR features offer absolute geometric constraints to resolve scale and orientation;
    \item Online estimation of $\mathbf{g}^G$ removes dependence on a priori gravity direction, enhancing robustness in non-horizontal scenarios;
    \item With sufficient non-degenerate features, the observability matrix has full rank.
\end{itemize}

This aligns with the theoretical analysis in \cite{fast_lio}, confirming that the state can be uniquely determined in typical operating conditions.

\section{Iterated Extended Kalman Filter (IEKF)}
\label{sec:iekf}

Assume the number of feature points is $m$, each is sampled at time $\rho_j \in (t_{k-1}, t_k]$ and is denoted as ${}^{L_j}\mathbf{p}_{f_j}$, where $L_j$ is the LiDAR local frame at the time $\rho_j$. During a LiDAR scan, there are also multiple IMU measurements, each sampled at time $\tau_i \in [t_{k-1}, t_k]$ with the respective state $\mathbf{x}_i$ as in \eqref{eq:discrete_state}. Notice that the last LiDAR feature point is the end of a scan, i.e., $\rho_m = t_k$, while the IMU measurements may not necessarily be aligned with the start or end of the scan.

To estimate the states in the state formulation \eqref{eq:discrete_state}, we use an iterated extended Kalman filter. Assume the optimal state estimate of the last LiDAR scan at $t_{k-1}$ is $\bar{\mathbf{x}}_{k-1}$ with covariance matrix $\bar{\mathbf{P}}_{k-1}$. Then $\bar{\mathbf{P}}_{k-1}$ represents the covariance of the random error state vector defined below:
\begin{equation*}
    \widetilde{\mathbf{x}}_{k-1} \doteq \mathbf{x}_{k-1} \boxminus \bar{\mathbf{x}}_{k-1} = \begin{bmatrix}
        \delta\boldsymbol{\theta}^T & {}^G\widetilde{\mathbf{p}}_I^T & {}^G\widetilde{\mathbf{v}}_I^T & \widetilde{\mathbf{b}}_{\boldsymbol{\omega}}^T & \widetilde{\mathbf{b}}_{\mathbf{a}}^T & {}^G\widetilde{\mathbf{g}}^T
    \end{bmatrix}^T,
    \label{eq:error_state}
\end{equation*}
where $\delta\boldsymbol{\theta} = \mathrm{Log}({}^G\bar{\mathbf{R}}_I^T {}^G\mathbf{R}_I)$ is the attitude error and the rests are standard additive errors (i.e., the error in the estimate $\widetilde{\mathbf{x}}$ of a quantity $\mathbf{x}$ is $\widetilde{\mathbf{x}} = \mathbf{x} - \bar{\mathbf{x}}$). 
Intuitively, the attitude error $\delta\boldsymbol{\theta}$ describes the (small) deviation between the true and the estimated attitude. The main advantage of this error definition is that it allows us to represent the attitude uncertainty by the $3\times3$ covariance matrix $\mathbb{E}\left\{ \delta\boldsymbol{\theta} \delta\boldsymbol{\theta}^T \right\}$. Since the attitude has 3 degree of freedom (DOF), this is a minimal representation.

\subsection{Forward Propagation}
\label{subsec:forward_propagation}
\begin{figure}[htbp]
    \centering
    \includegraphics[width=0.6\textwidth]{media/forward_backword.pdf}
    \caption{The forward and backward propagation.}
    \label{fig:forward_propagation}
\end{figure}
The forward propagation is performed once receiving an IMU input. More specifically, the state is propagated following \eqref{eq:discrete_state} by setting the process noise $\mathbf{w}_i$ to zero:
\begin{equation}
    \widehat{\mathbf{x}}_{i+1} = \widehat{\mathbf{x}}_i \boxplus (\Delta t \mathbf{f}(\widehat{\mathbf{x}}_i, \mathbf{u}_i, \mathbf{0})); \quad \widehat{\mathbf{x}}_0 = \bar{\mathbf{x}}_{k-1}.
    \label{eq:forward_propagation}
\end{equation}
where $\Delta t = \tau_{i+1} - \tau_i$. To propagate the covariance, we use the error state dynamic model obtained below:
\begin{align}
    \widetilde{\mathbf{x}}_{i+1} &= \mathbf{x}_{i+1} \boxminus \widehat{\mathbf{x}}_{i+1} \nonumber \\
    &= (\mathbf{x}_i \boxplus \Delta t \mathbf{f}(\mathbf{x}_i, \mathbf{u}_i, \mathbf{w}_i)) \boxminus (\widehat{\mathbf{x}}_i \boxplus \Delta t \mathbf{f}(\widehat{\mathbf{x}}_i, \mathbf{u}_i, \mathbf{0})) \label{eq:error_dynamics} \\
    &\simeq \mathbf{F}_{\widetilde{\mathbf{x}}} \widetilde{\mathbf{x}}_i + \mathbf{F}_{\mathbf{w}} \mathbf{w}_i.
    \label{eq:error_dynamics_lin}
\end{align}

The matrix $\mathbf{F}_{\widetilde{\mathbf{x}}}$ and $\mathbf{F}_{\mathbf{w}}$ in \eqref{eq:error_dynamics_lin} is computed following Appendix~\ref{appendix:derivation}.
\begin{equation}
    \begin{gathered}
        \mathbf{F}_{\mathbf{x}}=
        \begin{bmatrix}
            \mathrm{Exp}\left(-\widehat{\boldsymbol{\omega}}_{i}\Delta t\right)&\mathbf{0}&\mathbf{0}&-\mathbf{A}(\widehat{\boldsymbol{\omega}}_{i}\Delta t)^{T}\Delta t&\mathbf{0}&\mathbf{0}\\
            \mathbf{0}&\mathbf{I}&\mathbf{I}\Delta t&\mathbf{0}&\mathbf{0}\\
            -^{G}\widehat{\mathbf{R}}_{I_{i}}\lfloor\widehat{\mathbf{a}}_{i}\rfloor_{\wedge}\Delta t&\mathbf{0}&\mathbf{I}&\mathbf{0}&-^{G}\widehat{\mathbf{R}}_{I_{i}}\Delta t&\mathbf{I}\Delta t\\
            \mathbf{0}&\mathbf{0}&\mathbf{0}&\mathbf{I}&\mathbf{0}&\mathbf{0}\\
            \mathbf{0}&\mathbf{0}&\mathbf{0}&\mathbf{0}&\mathbf{I}&\mathbf{0}\\
            \mathbf{0}&\mathbf{0}&\mathbf{0}&\mathbf{0}&\mathbf{0}&\mathrm{I}\end{bmatrix},\\
        \mathbf{F}_{\mathbf{w}}=
        \begin{bmatrix}
            -\mathbf{A}\left(\widehat{\boldsymbol{\omega}}_{i}\Delta t\right)^{T}\Delta t&\mathbf{0}&\mathbf{0}&\mathbf{0}\\
            \mathbf{0}&\mathbf{0}&\mathbf{0}&\mathbf{0}\\\mathbf{0}&-^{G}\widehat{\mathbf{R}}_{I_{i}}\Delta t&\mathbf{0}&\mathbf{0}\\
            \mathbf{0}&\mathbf{0}&\mathbf{I}\Delta t&\mathbf{0}\\
            \mathbf{0}&\mathbf{0}&\mathbf{0}&\mathbf{I}\Delta t\\
            \mathbf{0}&\mathbf{0}&\mathbf{0}&\mathbf{0}\end{bmatrix}
    \label{eq:linearized_matrices}
    \end{gathered}
\end{equation}
The result is shown in \eqref{eq:linearized_matrices}, where $\widetilde{\boldsymbol{\omega}}_i = \boldsymbol{\omega}_{m_i} - \widehat{\mathbf{b}}_{\omega_i}$, $\widetilde{\mathbf{a}}_i = \mathbf{a}_{m_i} - \widehat{\mathbf{b}}_{a_i}$, and $\mathbf{A}(\mathbf{u})^{-1}$ follows the same definition in \cite{Bullo_Murray_1995} as below:
\begin{equation}
    \begin{split}
        \mathbf{A}(\mathbf{u})^{-1} = \mathbf{I} - \frac{1}{2} [\mathbf{u}]_\wedge + \left(1 - \alpha(\|\mathbf{u}\|)\right) \frac{\mathbf{u}\mathbf{u}^T}{\|\mathbf{u}\|^2},\\
        \alpha(m) = \frac{m}{2} \cot\left(\frac{m}{2}\right) = \frac{m}{2} \frac{\cos(m/2)}{\sin(m/2)}.
    \end{split}
    \label{eq:attitude_matrix_inverse}
\end{equation}

Denoting the covariance of white noises $\mathbf{w}$ as $\mathbf{Q}$, then the propagated covariance $\widehat{\mathbf{P}}_i$ can be computed iteratively following the below equation:
\begin{equation}
    \widehat{\mathbf{P}}_{i+1} = \mathbf{F}_{\widetilde{\mathbf{x}}} \widehat{\mathbf{P}}_i \mathbf{F}_{\widetilde{\mathbf{x}}}^T + \mathbf{F}_{\mathbf{w}} \mathbf{Q} \mathbf{F}_{\mathbf{w}}^T; \quad \widehat{\mathbf{P}}_0 = \bar{\mathbf{P}}_{k-1}.
    \label{eq:covariance_propagation}
\end{equation}

The propagation continues until reaching the end time of a new scan at $t_k$ where the propagated state and covariance are denoted as $\widehat{\mathbf{x}}_k, \widehat{\mathbf{P}}_k$. Then $\widehat{\mathbf{P}}_k$ represents the covariance of the error between the ground-truth state $\mathbf{x}_k$ and the state propagation $\widehat{\mathbf{x}}_k$ (i.e., $\mathbf{x}_k \boxminus \widehat{\mathbf{x}}_k$).

\subsection{Backward Propagation and Motion Compensation}
\label{subsec:backward_propagation}
When the points accumulation time interval is reached at time $t_k$, the new scan of feature points should be fused with the propagated state $\widehat{\mathbf{x}}_k$ and covariance $\widehat{\mathbf{P}}_k$ to produce an optimal state update. However, although the new scan is at time $t_k$, the feature points are measured at their respective sampling time $\rho_j \leq t_k$ 
(see Fig.~\ref{fig:forward_propagation})
, causing a mismatch in the body frame of reference.

To compensate for the relative motion (i.e., motion distortion) between time $\rho_j$ and time $t_k$, we propagate \eqref{eq:discrete_state} backward as:
\[
    \widetilde{\mathbf{x}}_{j-1} = \widetilde{\mathbf{x}}_j \boxplus (-\Delta t \mathbf{f}(\widetilde{\mathbf{x}}_j, \mathbf{u}_j, \mathbf{0})),
\]
starting from zero pose and rests states (e.g., velocity and bias) from $\widehat{\mathbf{x}}_k$. The backward propagation is performed at the frequency of feature point, which is usually much higher than the IMU rate. For all the feature points sampled between two IMU measurements, we use the left IMU measurement as the input in the back propagation. Furthermore, noticing that the last three block elements (corresponding to the gyro bias, accelerometer bias, and extrinsic) of $\mathbf{f}(\mathbf{x}_j, \mathbf{u}_j, \mathbf{0})$ (see \eqref{eq:discrete_dynamics}) are zeros, the back propagation can be reduced to:
\begin{equation}
    \begin{aligned}
        {}^{I_k}\check{\mathbf{p}}_{I_{j-1}} &= {}^{I_k}\check{\mathbf{p}}_{I_j} - {}^{I_k}\check{\mathbf{v}}_{I_j} \Delta t, & \text{s.f. } {}^{I_k}\check{\mathbf{p}}_{I_m} &= \mathbf{0}; \\
        {}^{I_k}\check{\mathbf{v}}_{I_{j-1}} &= {}^{I_k}\check{\mathbf{v}}_{I_j} - {}^{I_k}\check{\mathbf{R}}_{I_j} (\mathbf{a}_{m_{i-1}} - \widehat{\mathbf{b}}_{a_i}) \Delta t - {}^{I_k}\check{\mathbf{g}}_k \Delta t, & \text{s.f. } {}^{I_k}\check{\mathbf{v}}_{I_m} &= {}^{G}\widehat{\mathbf{R}}_{I_k}^T {}^{G}\widehat{\mathbf{v}}_{I_k}, \quad {}^{I_k}\check{\mathbf{g}}_k = {}^{G}\widehat{\mathbf{R}}_{I_k}^T {}^{G}\widehat{\mathbf{g}}_k; \\
        {}^{I_k}\check{\mathbf{R}}_{I_{j-1}} &= {}^{I_k}\check{\mathbf{R}}_{I_j} \mathrm{Exp}((\widehat{\mathbf{b}}_{\omega_k} - \boldsymbol{\omega}_{m_{i-1}}) \Delta t), & \text{s.f. } {}^{I_k}\check{\mathbf{R}}_{I_m} &= \mathbf{I}.
    \end{aligned}
    \label{eq:backward_propagation}
\end{equation}
where $\rho_{j-1} \in [\tau_{i-1}, \tau_i)$, $\Delta t = \rho_j - \rho_{j-1}$, and s.f. means “starting from”.

The backward propagation will produce a relative pose between time $\rho_j$ and the scan-end time $t_k$: ${}^{I_k}\check{\mathbf{T}}_{I_j} = \begin{bmatrix} {}^{I_k}\check{\mathbf{R}}_{I_j} & {}^{I_k}\check{\mathbf{p}}_{I_j} \end{bmatrix}$. This relative pose enables us to project the local measurement ${}^{L_j}\mathbf{p}_{f_j}$ to scan-end measurement ${}^{L_k}\mathbf{p}_{f_j}$ as follows 
(see Fig.~\ref{fig:forward_propagation}):
\begin{equation}
    {}^{L_k}\mathbf{p}_{f_j} = {}^{I}\mathbf{T}_L^{-1} {}^{I_k}\check{\mathbf{T}}_{I_j} {}^{I}\mathbf{T}_L {}^{L_j}\mathbf{p}_{f_j}.
    \label{eq:scan_end_projection}
\end{equation}
where ${}^{I}\mathbf{T}_L$ is the known extrinsic (see Section~\ref{subsec:system_description}). Then the projected point ${}^{L_k}\mathbf{p}_{f_j}$ is used to construct a residual in the following section.

\subsection{Residual computation}
\label{subsec:residual_computation}

With the motion compensation in \eqref{eq:scan_end_projection}, we can view the scan of feature points $\{{}^{L_k}\mathbf{p}_{f_j}\}$, all sampled at the same time $t_k$, and use it to construct the residual. Assume the current iteration of the iterated Kalman filter is $\kappa$, and the corresponding state estimate is $\widehat{\mathbf{x}}_k^\kappa$. When $\kappa = 0$, $\widehat{\mathbf{x}}_k^\kappa = \widehat{\mathbf{x}}_k$, the predicted state from the propagation in \eqref{eq:forward_propagation}. Then, the feature points $\{{}^{L_k}\mathbf{p}_{f_j}\}$ can be transformed to the global frame as follows:

\begin{equation}
    {}^G\widehat{\mathbf{p}}_{f_j}^\kappa = {}^G\widehat{\mathbf{T}}_{I_k}^{\kappa} {}^I\mathbf{T}_L {}^{L_k}\mathbf{p}_{f_j}; \quad j = 1, \cdots, m.
    \label{eq:global_projection}
\end{equation}

For each LiDAR feature point, the closest plane or edge defined by its nearby feature points in the map is assumed to be where the point truly belongs to. The residual is defined as the distance between the feature point's estimated global frame coordinate ${}^G\widehat{\mathbf{p}}_{f_j}^\kappa$ and the nearest plane (or edge) in the map. Denoting $\mathbf{u}_j$ the normal vector (or edge orientation) of the corresponding plane (or edge), on which lying a point ${}^G\mathbf{q}_j$, then the residual $\mathbf{z}_j^\kappa$ is computed as:

\begin{equation}
    \mathbf{z}_j^\kappa = \mathbf{G}_j \left( {}^G\widehat{\mathbf{p}}_{f_j}^\kappa - {}^G\mathbf{q}_j \right)
    \label{eq:residual}
\end{equation}

where $\mathbf{G}_j = \mathbf{u}_j^T$ for planar features and $\mathbf{G}_j = [\mathbf{u}_j]_\times$ for edge features. The computation of the $\mathbf{u}_j$ and the search of nearby points in the map, which define the corresponding plane or edge, is achieved by building a KD-tree of the points in the most recent map \cite{lin2020loam}. 

\subsection{Iterated State Update}
\label{subsec:iterated_update}

To fuse the residual $\mathbf{z}_j^\kappa$ computed in \eqref{eq:residual} with the state prediction $\widehat{\mathbf{x}}_k$ and covariance $\widehat{\mathbf{P}}_k$ propagated from the IMU data, we need to linearize the measurement model that relates the residual $\mathbf{z}_j^\kappa$ to the ground-truth state $\mathbf{x}_k$ and measurement noise. The measurement noise originates from the LiDAR ranging and beam-directing noise ${}^{L_j}\mathbf{n}_{f_j}$ when measuring the point ${}^{L_j}\mathbf{p}_{f_j}$. Removing this noise from the point measurement ${}^{L_j}\mathbf{p}_{f_j}$ leads to the true point location:

\begin{equation}
    {}^{L_j}\mathbf{p}_{f_j}^{\text{gt}} = {}^{L_j}\mathbf{p}_{f_j} - {}^{L_j}\mathbf{n}_{f_j}.
    \label{eq:true_point}
\end{equation}

This true point, after projecting to the frame $L_k$ via \eqref{eq:scan_end_projection} and then to the global frame with the ground-truth state $\mathbf{x}_k$ (i.e., pose), should lie exactly on the plane (or edge) in the map. That is, plugging \eqref{eq:true_point} into \eqref{eq:scan_end_projection}, then into \eqref{eq:global_projection}, and further into \eqref{eq:residual} should result in zero, i.e.,

\begin{equation}
    \mathbf{0} = \mathbf{h}_j(\mathbf{x}_k, {}^{L_j}\mathbf{n}_{f_j}) = \mathbf{G}_j \left( {}^G\mathbf{T}_{I_k}^\kappa {}^I\mathbf{T}_L {}^{L_k}\mathbf{T}_{L_j} ({}^{L_j}\mathbf{p}_{f_j} - {}^{L_j}\mathbf{n}_{f_j}) - {}^G\mathbf{q}_j \right).
    \label{eq:zero_residual}
\end{equation}

Now,we find out the relation between the state $\widehat{\mathbf{x}}_k^\kappa$ and the residual $\mathbf{z}_j^\kappa$.
Note that the $\mathbf{h}_j$ is nonlinear with respect to the state $\mathbf{x}_k$,then we can approximate the above equation by its first-order Taylor expansion made at $\widehat{\mathbf{x}}_k^\kappa$ :

\begin{align}
    \mathbf{0} &= \mathbf{h}_j \left( \mathbf{x}_k, {}^{L_j}\mathbf{n}_{f_j} \right) \simeq \mathbf{h}_j \left( \widehat{\mathbf{x}}_k^\kappa, \mathbf{0} \right) + \mathbf{H}_j^\kappa \widetilde{\mathbf{x}}_k^\kappa + \mathbf{v}_j \nonumber \\
    &= \mathbf{z}_j^\kappa + \mathbf{H}_j^\kappa \widetilde{\mathbf{x}}_k^\kappa + \mathbf{v}_j,
    \label{eq:linearized_measurement}
\end{align}

where $\widetilde{\mathbf{x}}_k^\kappa = \mathbf{x}_k \boxminus \widehat{\mathbf{x}}_k^\kappa$ is the error state, $\mathbf{H}_j^\kappa$ is the Jacobian matrix of $\mathbf{h}_j(\widehat{\mathbf{x}}_k^\kappa \boxplus \widetilde{\mathbf{x}}_k^\kappa, {}^{L_j}\mathbf{n}_{f_j})$ with respect to $\widetilde{\mathbf{x}}_k^\kappa$, evaluated at zero, and $\mathbf{v}_j \sim \mathcal{N}(\mathbf{0}, \mathbf{R}_j)$ comes from the raw measurement noise ${}^{L_j}\mathbf{n}_{f_j}$.

Notice that the prior distribution of $\mathbf{x}_k$ obtained from the forward propagation in Section~\ref{subsec:forward_propagation} is for:
\begin{equation}
    \mathbf{x}_k \boxminus \widehat{\mathbf{x}}_k = (\widetilde{\mathbf{x}}_k^\kappa \boxplus \widehat{\mathbf{x}}_k^\kappa) \boxminus \widehat{\mathbf{x}}_k = \widetilde{\mathbf{x}}_k^\kappa \boxplus (\widehat{\mathbf{x}}_k^\kappa \boxminus \widehat{\mathbf{x}}_k) = \widetilde{\mathbf{x}}_k^\kappa + \mathbf{J}^\kappa \widetilde{\mathbf{x}}_k^\kappa,
    \label{eq:prior_error}
\end{equation}
where $\mathbf{J}^\kappa$ is the partial differentiation of $(\widehat{\mathbf{x}}_k^\kappa \boxplus \widetilde{\mathbf{x}}_k^\kappa) \boxminus \widehat{\mathbf{x}}_k$ with respect to $\widetilde{\mathbf{x}}_k^\kappa$ evaluated at zero:
\begin{equation}
    \mathbf{J}^\kappa = 
    \begin{bmatrix}
        \mathbf{A}\left( {}^G\widehat{\mathbf{R}}_{I_k}^\kappa \boxminus {}^G\widehat{\mathbf{R}}_{I_k} \right)^{-T} & \mathbf{0}_{3\times15} \\
        \mathbf{0}_{15\times3} & \mathbf{I}_{15\times15}
    \end{bmatrix},
    \label{eq:jacobian_J}
\end{equation}
and $\mathbf{A}(\cdot)^{-1}$ is defined in \eqref{eq:attitude_matrix_inverse}. $\mathbf{J}^\kappa$ can be obtained using the conclusion in Appendix \ref{appendix:derivation}. For the first iteration (i.e., the case of extended Kalman filter), $\widehat{\mathbf{x}}_k^\kappa = \widehat{\mathbf{x}}_k$, then $\mathbf{J}^\kappa = \mathbf{I}$.

Combining the prior in \eqref{eq:prior_error} with the posteriori distribution from \eqref{eq:linearized_measurement} yields the maximum a-posteriori estimate (MAP):
\begin{equation}
    \min_{\widetilde{\mathbf{x}}_k^\kappa} \left( \| \mathbf{x}_k \boxminus \widehat{\mathbf{x}}_k \|_{\widehat{\mathbf{P}}_k^{-1}}^2 + \sum_{j=1}^{m} \| \mathbf{z}_j^\kappa + \mathbf{H}_j^\kappa \widetilde{\mathbf{x}}_k^\kappa \|_{\mathbf{R}_j^{-1}}^2 \right),
    \label{eq:map_estimation}
\end{equation}
where $\| \mathbf{x} \|_{\mathbf{M}}^2 = \mathbf{x}^T \mathbf{M} \mathbf{x}$. Substituting the linearization of the prior in \eqref{eq:prior_error} into \eqref{eq:map_estimation} and optimizing the resultant quadratic cost leads to the standard iterated Kalman filter \cite{thrun2005probabilistic}, which can be computed as follows (to simplify the notation, let $\mathbf{H} = [\mathbf{H}_1^{\kappa T}, \cdots, \mathbf{H}_m^{\kappa T}]^T$, $\mathbf{R} = \mathrm{diag}(\mathbf{R}_1, \cdots, \mathbf{R}_m)$, $\mathbf{P} = (\mathbf{J}^\kappa)^{-1} \widehat{\mathbf{P}}_k (\mathbf{J}^\kappa)^{-T}$, and $\mathbf{z}_k^\kappa = [ {\mathbf{z}_1^{\kappa T}}, \cdots, {\mathbf{z}_m^{\kappa T}} ]^T$):

\begin{equation}
    \mathbf{K} = \mathbf{P} \mathbf{H}^T (\mathbf{H} \mathbf{P} \mathbf{H}^T + \mathbf{R})^{-1},
    \label{eq:kalman_gain_standard}
\end{equation}
\begin{equation}
    \widehat{\mathbf{x}}_k^{\kappa+1} = \widehat{\mathbf{x}}_k^\kappa \boxplus \left( -\mathbf{K} \mathbf{z}_k^\kappa - (\mathbf{I} - \mathbf{K} \mathbf{H}) (\mathbf{J}^\kappa)^{-1} (\widehat{\mathbf{x}}_k^\kappa \boxminus \widehat{\mathbf{x}}_k) \right).
    \label{eq:state_update}
\end{equation}

The updated estimate $\widehat{\mathbf{x}}_k^{\kappa+1}$ is then used to compute the residual in Section~\ref{subsec:residual_computation} and repeat the process until convergence (i.e., $\| \widehat{\mathbf{x}}_k^{\kappa+1} \boxminus \widehat{\mathbf{x}}_k^\kappa \| < \epsilon$). After convergence, the optimal state estimation and covariance is:
\begin{equation}
    \bar{\mathbf{x}}_k = \widehat{\mathbf{x}}_k^{\kappa+1}, \quad \bar{\mathbf{P}}_k = (\mathbf{I} - \mathbf{K} \mathbf{H}) \mathbf{P}.
    \label{eq:final_estimate}
\end{equation}

\subsection{Kalman gain computation}
A problem with the commonly used Kalman gain form is that it requires to invert the matrix $\mathbf{H}\mathbf{P}\mathbf{H}^T+\mathbf{R}$ which is in the dimension of the measurements. In practice, the number of LiDAR feature points are very large in number. Thus, inverting a matrix of this size is prohibitive.

In fact, if directly solving \eqref{eq:map_estimation}, we can obtain the same solution in \eqref{eq:kalman_gain_standard} but with a new form of Kalman gain shown below:

\begin{equation}
    \mathbf{K} = \left( \mathbf{H}^T \mathbf{R}^{-1} \mathbf{H} + \mathbf{P}^{-1} \right)^{-1} \mathbf{H}^T \mathbf{R}^{-1}.
    \label{eq:new_kalman_gain}
\end{equation}

Next we prove that the two forms of Kalman gains are indeed equivalent. 

Based on the matrix inverse lemma \cite{woodbury}, we can get:
\begin{equation}
    (\mathbf{P}^{-1} + \mathbf{H}^T \mathbf{R}^{-1} \mathbf{H})^{-1} = \mathbf{P} - \mathbf{P} \mathbf{H}^T (\mathbf{H} \mathbf{P} \mathbf{H}^T + \mathbf{R})^{-1} \mathbf{H} \mathbf{P}.
    \label{eq:matrix_inverse_lemma}
\end{equation}

Substituting the above into \eqref{eq:new_kalman_gain}, we can get:
\begin{align}
    \mathbf{K} &= (\mathbf{H}^T \mathbf{R}^{-1} \mathbf{H} + \mathbf{P}^{-1})^{-1} \mathbf{H}^T \mathbf{R}^{-1} \nonumber \\
               &= \mathbf{P} \mathbf{H}^T \mathbf{R}^{-1} - \mathbf{P} \mathbf{H}^T (\mathbf{H} \mathbf{P} \mathbf{H}^T + \mathbf{R})^{-1} \mathbf{H} \mathbf{P} \mathbf{H}^T \mathbf{R}^{-1}.
    \label{eq:kalman_gain_intermediate}
\end{align}

Now note that $\mathbf{H} \mathbf{P} \mathbf{H}^T \mathbf{R}^{-1} = (\mathbf{H} \mathbf{P} \mathbf{H}^T + \mathbf{R}) \mathbf{R}^{-1} - \mathbf{I}$. Substituting it into the above, we can get the standard Kalman gain formula in \eqref{eq:kalman_gain_standard}, as shown below:
\begin{align}
    \mathbf{K} &= \mathbf{P} \mathbf{H}^T \mathbf{R}^{-1} - \mathbf{P} \mathbf{H}^T \mathbf{R}^{-1} + \mathbf{P} \mathbf{H}^T (\mathbf{H} \mathbf{P} \mathbf{H}^T + \mathbf{R})^{-1} \nonumber \\
               &= \mathbf{P} \mathbf{H}^T (\mathbf{H} \mathbf{P} \mathbf{H}^T + \mathbf{R})^{-1}.
    \label{eq:standard_kalman_gain}
\end{align}

Since the LiDAR measurements are independent, the covariance matrix $\mathbf{R}$ is (block) diagonal and hence the new formula only requires to invert two matrices both in the dimension of state instead of measurements. The new formula greatly saves the computation as the state dimension is usually much lower than measurements in LIO (e.g., more than 1,000 effective feature points in a scan for 10 Hz scan rate while the state dimension is only 18).

This formulation reduces the computational complexity from $\mathcal{O}(m^3)$ to $\mathcal{O}(n^3)$, where $m$ is the number of measurements (feature points) and $n$ is the state dimension ($n=18$ in our case). This efficiency is critical for real-time operation on resource-constrained platforms.This could be proved in experimental section.The state estimation is summarized in Algorithm 1.
\newpage
\begin{algorithm}
    \caption{State Estimation}
    \label{alg:state_estimation}
    \textbf{Input} : Last optimal estimation $\bar{\mathbf{x}}_{k-1}$ and $\bar{\mathbf{P}}_{k-1}$, \\
    \hspace*{1.5em} IMU inputs $(\mathbf{a}_m, \boldsymbol{\omega}_m)$ in current scan; \\
    \hspace*{1.5em} LiDAR feature points ${}^{L_j}\mathbf{p}_{f_j}$ in current scan.
    
    \begin{algorithmic}[htbp]
        \State Forward propagation to obtain state prediction $\widehat{\mathbf{x}}_k$ via \eqref{eq:forward_propagation} and covariance prediction $\widehat{\mathbf{P}}_k$ via \eqref{eq:covariance_propagation};
        \State Backward propagation to obtain ${}^{L_k}\mathbf{p}_{f_j}$ via \eqref{eq:backward_propagation}, \eqref{eq:scan_end_projection};
        \State $\kappa = -1$, $\widehat{\mathbf{x}}_k^{\kappa=0} = \widehat{\mathbf{x}}_k$;
        \Repeat
            \State $\kappa = \kappa + 1$;
            \State Compute $\mathbf{J}^\kappa$ via \eqref{eq:jacobian_J} and $\mathbf{P} = (\mathbf{J}^\kappa)^{-1} \widehat{\mathbf{P}}_k (\mathbf{J}^\kappa)^{-T}$;
            \State Compute residual $\mathbf{z}_j^\kappa$ \eqref{eq:residual} and Jacobian $\mathbf{H}_j^\kappa$ \eqref{eq:linearized_measurement};
            \State Compute the state update $\widehat{\mathbf{x}}_k^{\kappa+1}$ via \eqref{eq:state_update} with the Kalman gain $\mathbf{K}$ from \eqref{eq:new_kalman_gain};
        \Until{$\| \widehat{\mathbf{x}}_k^{\kappa+1} \boxminus \widehat{\mathbf{x}}_k^\kappa \| < \epsilon$};
        \State $\bar{\mathbf{x}}_k = \widehat{\mathbf{x}}_k^{\kappa+1}$; $\bar{\mathbf{P}}_k = (\mathbf{I} - \mathbf{K} \mathbf{H}) \mathbf{P}$.
    \end{algorithmic}
    
    \textbf{Output}: Current optimal estimation $\bar{\mathbf{x}}_k$ and $\bar{\mathbf{P}}_k$.
\end{algorithm}

\section{Implementation and Experimental Setup}
\begin{itemize}
    \item \textbf{Language}: Python 3.10 with \texttt{NumPy}, \texttt{Matplotlib}, \texttt{filterpy}
    \item \textbf{Simulation}: 100 time steps, 50 Monte Carlo runs
    \item \textbf{Noise}: $\mathbf{Q} = \mathrm{diag}([0.1, 0.1, 0.01, 0.001, 0.001])$, $\mathbf{R} = \mathrm{diag}([0.5, 0.01, 0.1])$
    \item \textbf{Metrics}: MSE, RMSE, computation time
\end{itemize}

\section{Results and Analysis}
\begin{figure}[H]
\centering
% \includegraphics[width=0.8\linewidth]{figures/trajectory.pdf}
\caption{True vs. estimated trajectory (UKF vs. EKF). UKF better captures sharp turns.}
\label{fig:traj}
\end{figure}

% Table~\ref{tab:results} summarizes performance:
\begin{table}[H]
\centering
\caption{Estimation Error Comparison}
\label{tab:results}
\begin{tabular}{lcc}
\toprule
Method & MSE ($\times 10^{-2}$) & Avg. Time (ms) \\
\midrule
EKF & 3.42 & 1.8 \\
UKF & 2.67 & 3.5 \\
\bottomrule
\end{tabular}
\end{table}

The UKF reduces MSE by 22\% at the cost of higher computation. Sensitivity analysis shows UKF maintains accuracy even when $\mathbf{R}$ increases by 200\%.

\section{Conclusion}
\label{sec:conclusion}
\subsection{Main Contributions}
\begin{itemize}
    \item \textbf{Complete Reproduction of the FAST-LIO Core Workflow}: 
    
    Implemented a tightly-coupled state estimation algorithm based on Iterative Extended Kalman Filter (IEKF), including IMU forward propagation, LiDAR point cloud backward propagation for motion distortion compensation, residual construction, and iterative updates.
    \item \textbf{Validation of Computational Efficiency Optimization}: 
    
    Experimentally demonstrated that the new form of Kalman gain based on the Woodbury matrix identity \eqref{eq:new_kalman_gain} reduces computational complexity from $\mathcal{O}(m^3)$ to $\mathcal{O}(n^3)$, enabling real-time processing of over a thousand feature points on resource-constrained platforms (e.g., UAV onboard computers).
    \item \textbf{Robustness Validation Across Scenarios}: 
    
    Achieved sub-meter localization accuracy (RMSE = 0.210 m) in indoor (feature-degraded) and outdoor (high-dynamic) scenarios of the M3DGR dataset, demonstrating strong robustness to lighting changes, geometric degradation, and motion distortion.
\end{itemize}

\subsection{Algorithm Advantages}
Compared to traditional loosely-coupled or optimization-based methods, the FAST-LIO framework offers the following significant advantages:
\begin{itemize}
    \item \textbf{Enhanced Robustness through Tight Coupling}: Directly integrates raw LiDAR feature points with IMU states, avoiding the degradation issues of loosely-coupled methods caused by feature loss.
    \item \textbf{Precise Motion Distortion Compensation}: Utilizes a backward propagation mechanism to independently align time and pose for each LiDAR point, significantly improving point cloud geometric consistency.
    \item \textbf{Real-Time Efficiency}: Thanks to the dimensional optimization of the Kalman gain, the system can stably operate at a frequency of 10 Hz without blocking.
    \item \textbf{Minimal State Representation}: Defines error states on the Lie group manifold, using minimal parameterization (3 DoF) for attitude uncertainty, enhancing filter stability.
\end{itemize}

\subsection{Limitations and Future Work}
Nevertheless, several limitations remain. The method assumes high-frequency, low-noise IMU data; performance may degrade under severe vibration or with low-cost IMUs. Moreover, the absence of loop closure means long-term drift accumulates uncorrected, limiting scalability to large-scale mapping. The feature extraction module, while effective in structured environments, can still fail in textureless or highly repetitive scenes.

Looking ahead, future work could integrate FAST-LIO with a global optimization backend to enable loop closure and map refinement. Another promising direction is the fusion of visual or event-based sensors to enhance robustness in LiDAR-degenerate conditions. Additionally, deploying the system on embedded platforms with hardware-aware acceleration—such as fixed-point arithmetic or sparse matrix optimizations—would further broaden its applicability in real-world robotic systems. Ultimately, this project not only validates the theoretical elegance of FAST-LIO but also underscores the practical viability of tightly-coupled filtering for next-generation autonomous navigation.

\section{Project Management and Division of Work}
\label{sec:management}
This project was completed through the collaboration of three members, with specific responsibilities as follows:
\begin{itemize}
    \item Junda Wu: Responsible for overall project coordination and compilation of the theoretical section.
    \item Hongkang Cui: Responsible for experimental design and result analysis.
    \item Yongjian Liu: Responsible for writing the report documentation.
\end{itemize}

\appendix

\part*{Appendix}

\section{Derivation of the attitude kinematics \texorpdfstring{\eqref{rotation_dyn}}{(2c)}}
\label{appendix: rotation}

The rotation matrix from the global frame to the IMU frame can be represented as
\[
{}^G\mathbf{R}_I = [{}^G\mathbf{x}_I\quad {}^G\mathbf{y}_I\quad {}^G\mathbf{z}_I]
\]
Take the time derivative of the above equation and we obtain
\[{}^G\dot{\mathbf{R}}_I = [{}^G\dot{\mathbf{x}}_I\quad {}^G\dot{\mathbf{y}}_I\quad {}^G\dot{\mathbf{z}}_I] = [{}^G\boldsymbol{\omega}_I\times ^G\dot{\mathbf{x}}_I\quad {}^G\boldsymbol{\omega}_I\times {}^G\dot{\mathbf{y}}_I\quad {}^G\boldsymbol{\omega}_I\times {}^G\dot{\mathbf{z}}_I] = \lfloor{}^G\boldsymbol{\omega}_I\rfloor_\wedge {}^G{\mathbf{R}}_I\]

Since ${}^G\boldsymbol{\omega}_I = {}^G{\mathbf{R}}_I \boldsymbol{\omega}$ and $\lfloor\mathbf{R} \mathbf{p}\rfloor_\wedge = \mathbf{R}\lfloor\mathbf{p}\rfloor_\wedge\mathbf{R}^T$, we have
\[{}^G\dot{\mathbf{R}}_I = \lfloor{}^G\mathbf{R}_I \boldsymbol{\omega}\rfloor_\wedge {}^G{\mathbf{R}}_I = {}^G\mathbf{R}_I \lfloor\boldsymbol{\omega}\rfloor_\wedge {}^I\mathbf{R}_G {}^G\mathbf{R}_I = {}^G\mathbf{R}_I \lfloor\boldsymbol{\omega}\rfloor_\wedge.\]

\section{Derivation of the matrices in \texorpdfstring{\eqref{eq:linearized_matrices}}{(9)}}
\label{appendix:derivation}
From \eqref{eq:error_dynamics}, noting that ${\mathbf{x}}_i = \widehat{\mathbf{x}}_i \boxplus \widetilde{\mathbf{x}}_i$, we have
\[\widetilde{\mathbf{x}}_{i+1}=\left((\widehat{\mathbf{x}}_i\boxplus\widetilde{\mathbf{x}}_i)\boxplus\Delta t\mathbf{f}\left(\mathbf{x}_i,\mathbf{u}_i,\mathbf{w}_i\right)\right)\boxminus\left(\widehat{\mathbf{x}}_i\boxplus\Delta t\mathbf{f}\left(\widehat{\mathbf{x}}_i,\mathbf{u}_i,\mathbf{0}\right)\right)\]
Define
\[\mathbf{g}(\widetilde{\mathbf{x}}_i,\mathbf{w}_i)=\mathbf{f}(\mathbf{x}_i,\mathbf{u}_i,\mathbf{w}_i)\Delta t=\mathbf{f}(\widehat{\mathbf{x}}_i\boxplus\widetilde{\mathbf{x}}_i,\mathbf{u}_i,\mathbf{w}_i)\Delta t\]
we have
\[\widetilde{\mathbf{x}}_{i+1}=\underbrace{\left(\left(\widehat{\mathbf{x}}_i\boxplus\widetilde{\mathbf{x}}_i\right)\boxplus\mathbf{g}\left(\widetilde{\mathbf{x}}_i,\mathbf{w}_i\right)\right)\boxminus\left(\widehat{\mathbf{x}}_i\boxplus\mathbf{g}\left(\mathbf{0},\mathbf{0}\right)\right)}_{\mathbf{G}(\widetilde{\mathbf{x}}_i,\mathbf{g}(\widetilde{\mathbf{x}}_i,\mathbf{w}_i))}\]
Linearize the above equation at $\widetilde{\mathbf{x}}_{i}=\mathbf{0},\mathbf{w}_{i}=\mathbf{0}$, we obtain the expression for the matrices:
\[
\mathbf{F}_{\widetilde{\mathbf{x}}}=\left(\frac{\partial\mathbf{G}(\widetilde{\mathbf{x}}_{i},\mathbf{g}(\mathbf{0},\mathbf{0}))}{\partial\widetilde{\mathbf{x}}_{i}}+\frac{\partial\mathbf{G}(\mathbf{0},\mathbf{g}(\widetilde{\mathbf{x}}_{i},\mathbf{0}))}{\partial\mathbf{g}(\widetilde{\mathbf{x}}_{i},\mathbf{0})}\frac{\partial\mathbf{g}(\widetilde{\mathbf{x}}_{i},\mathbf{0})}{\partial\widetilde{\mathbf{x}}_{i}}\right)|_{\widetilde{\mathbf{x}}_{i}=\mathbf{0}}
\]
and
\[\mathbf{F}_{\mathbf{w}}=\left(\frac{\partial \mathbf{G}(\mathbf{0},\mathbf{g}(\mathbf{0},\mathbf{w}_{i}))}{\partial \mathbf{g}(\mathbf{0},\mathbf{w}_{i})}\frac{\partial \mathbf{g}(\mathbf{0},\mathbf{w}_{i})}{\partial \mathbf{w}_{i}}\right)|_{\mathbf{w}_{i}=\mathbf{0}}\]

To obtain the partial derivatives, we express $\mathbf{G}$ as follows
\[
\mathbf{G}=\left((\mathbf{a}\boxplus \mathbf{b})\boxplus \mathbf{c}\right)\boxminus \mathbf{d}
\]
where $\mathbf{G},\mathbf{b},\mathbf{c}\in\mathbb{R}^{18},\mathbf{a},\mathbf{d}\in SO(3)\times\mathbb{R}^{15}$.

There are two cases:

{\bf Case 1}: $\mathbf{a},\mathbf{b},\mathbf{c},\mathbf{d}\in\mathbb{R}^n,\quad\frac{\partial \mathbf{G}}{\partial \mathbf{b}}=\frac{\partial \mathbf{G}}{\partial \mathbf{c}}=\mathbf{I}_n$. This case is trivial, since the $\boxplus$ / $\boxminus$ operation on $\mathbb{R}^{n}$ are simply the general element-wise addition / subtraction.

{\bf Case 2}: $\mathbf{G},\mathbf{b},\mathbf{c}\in\mathbb{R}^{3},\mathbf{a},\mathbf{d}\in SO(3)$. In this case we have\[\frac{\partial \mathbf{G}}{\partial \mathbf{b}}=\mathbf{A}(\mathbf{G})^{-T}\mathrm{Exp}(-\mathbf{c})\mathbf{A}(\mathbf{b})^{T},\frac{\partial \mathbf{G}}{\partial \mathbf{c}}=\mathbf{A}(\mathbf{G})^{-T}\mathbf{A}(\mathbf{c})^{T}\]
where $\mathbf{A}(\cdot)$ is shown in \eqref{eq:attitude_matrix_inverse}.

\begin{proof}[Proof of Case 2]
    Since $\mathbf{G}=((\mathbf{a}\boxplus \mathbf{b})\boxplus \mathbf{c})\boxminus \mathbf{d} =\mathrm{Log}(\mathbf{d}^{-1}\cdot((\mathbf{a}\boxplus \mathbf{b})\boxplus \mathbf{c}))$, we have 
    \[\mathrm{Exp}(\mathbf{G})=\mathbf{d}^{-1}\cdot \mathbf{a}\cdot \mathrm{Exp}(\mathbf{b})\cdot \mathrm{Exp}(\mathbf{c}).\]
    Add perbutation to both sides and we have
    \[
    \mathrm{Exp}(\mathbf{G}+\Delta \mathbf{G})=\mathbf{d}^{-1}\cdot \mathbf{a}\cdot \mathrm{Exp}(\mathbf{b}+\Delta \mathbf{b})\cdot \mathrm{Exp}(\mathbf{c})
    \]
    Use Baker-Campbell-Hausdorff (BCH) formula, we have
    \[\mathrm{Exp}(\mathbf{G})\mathrm{Exp}(\mathbf{A}(\mathbf{G})^{T}\Delta \mathbf{G})=\mathbf{d}^{-1}\cdot \mathbf{a}\cdot \mathrm{Exp}(\mathbf{b})\mathrm{Exp}(\mathbf{A}(\mathbf{b})^{T}\Delta \mathbf{b})\cdot \mathrm{Exp}(\mathbf{c})\]
    Then
    \begin{align*}
        \mathrm{Exp}(\mathbf{A}(\mathbf{G})^{T}\Delta \mathbf{G})&=(\mathrm{Exp}(\mathbf{c}))^{-1}(\mathrm{Exp}(\mathbf{b}))^{-1} \mathbf{a}^{-1}\mathbf{d}\cdot \mathbf{d}^{-1}\cdot \mathbf{a}\cdot \mathrm{Exp}(\mathbf{b})\mathrm{Exp}(\mathbf{A}(\mathbf{b})^{T}\Delta \mathbf{b})\cdot \mathrm{Exp}(\mathbf{c})\\
        &=(\mathrm{Exp}(\mathbf{c}))^{T}\mathrm{Exp}(\mathbf{A}(\mathbf{b})^{T}\Delta \mathbf{b})\cdot \mathrm{Exp}(\mathbf{c})
    \end{align*}
    With the property that $\lfloor\mathbf{R} \mathbf{p}\rfloor_\wedge = \mathbf{R}\lfloor\mathbf{p}\rfloor_\wedge\mathbf{R}^T$, we have
    \[
    \mathrm{Exp}(\mathbf{A}(\mathbf{G})^{T}\Delta \mathbf{G}) = \mathrm{Exp}(\mathrm{Exp}(-\mathbf{c})\mathbf{A}(\mathbf{b})^{T}\Delta \mathbf{b})
    \]
    Taking the logarithm of both sides yields
    \[
    \frac{\Delta \mathbf{G}}{\Delta \mathbf{b}}=\mathbf{A}(\mathbf{G})^{-T}\mathrm{Exp}(-\mathbf{c})\mathbf{A}(\mathbf{b})^{T}.
    \]

    Similarly, add perbutation and we obtain
    \[
    \mathrm{Exp}(\mathbf{G}+\Delta \mathbf{G})= \mathbf{d}^{-1}\mathbf{a} \mathrm{Exp}(\mathbf{b})\mathrm{Exp}(\mathbf{c}+\Delta \mathbf{c})
    \]
    Use BCH formula again and we have
    \[
    \mathrm{Exp}(\mathbf{G})\mathrm{Exp}(\mathbf{A}(\mathbf{G})^{T}\Delta \mathbf{G})= \mathbf{d}^{-1}\mathbf{a} \mathrm{Exp}(\mathbf{b})\mathrm{Exp}(\mathbf{c})\mathrm{Exp}(\mathbf{A}(\mathbf{c})^T\Delta \mathbf{c})
    \]
    Then
    \begin{align*}
        \mathrm{Exp}(\mathbf{A}(\mathbf{G})^{T}\Delta \mathbf{G})&=(\mathrm{Exp}(\mathbf{c}))^{-1}(\mathrm{Exp}(\mathbf{b}))^{-1} \mathbf{a}^{-1}\mathbf{d}\cdot \mathbf{d}^{-1}\mathbf{a} \mathrm{Exp}(\mathbf{b})\mathrm{Exp}(\mathbf{c})\mathrm{Exp}(\mathbf{A}(\mathbf{c})^T\Delta \mathbf{c})\\
        &=\mathrm{Exp}(\mathbf{A}(\mathbf{c})^{T}\Delta \mathbf{c})
    \end{align*}
    Taking the logarithm of both sides yields
    \[
    \frac{\Delta \mathbf{G}}{\Delta \mathbf{c}}=\mathbf{A}(\mathbf{G})^{-T}\mathbf{A}(\mathbf{c})^{T}.\qedhere
    \]
\end{proof}

Then (Noticing that $\mathbf{A}(\mathbf{0})=\mathbf{I}$)
\begin{align}
    \frac{\partial\mathbf{G}(\widetilde{\mathbf{x}}_{i},\mathbf{g}(\mathbf{0},\mathbf{0}))}{\partial\widetilde{\mathbf{x}}_{i}} |_{\mathbf{\widetilde{x}}_i=\mathbf{0}}
    &= \begin{bmatrix}
   \mathbf{A}(\mathbf{0})^{-T}\mathrm{Exp}(-\mathbf{g}(\mathbf{0},\mathbf{0}))\mathbf{A}(\mathbf{0})^{T} & \mathbf{0}\\
   \mathbf{0} & \mathbf{I}_{15\times 15}
\end{bmatrix} \nonumber\\
&= \begin{bmatrix}\mathrm{Exp}(-\mathbf{f}(\mathbf{\widehat{x}},\mathbf{u},\mathbf{0})\Delta t)&\mathbf{0}\\\mathbf{0}&\mathbf{I}_{15\times15}\end{bmatrix}\nonumber\\
&=\begin{bmatrix}\mathrm{Exp}(-(\boldsymbol{\omega}_{m}-\mathbf{\widehat{b}}_{\boldsymbol{\omega}})\Delta{t})&\mathbf{0}\\\mathbf{0}&\mathbf{I}_{15\times15}\end{bmatrix}
\label{eq:partialGpartialx}
\end{align}
and
\begin{align}
    \frac{\partial \mathbf{G}(\mathbf{0},\mathbf{g}(\mathbf{0},\mathbf{w}_{i}))}{\partial \mathbf{g}(\mathbf{0},\mathbf{w}_{i})}|_{\mathbf{w}_i=\mathbf{0}} = \frac{\partial\mathbf{G}(\mathbf{0},\mathbf{g}(\widetilde{\mathbf{x}}_{i},\mathbf{0}))}{\partial\mathbf{g}(\widetilde{\mathbf{x}}_{i},\mathbf{0})}|_{\mathbf{\widetilde{x}}_i=\mathbf{0}} &=
    \begin{bmatrix}\mathbf{A(0)}^{-T}\mathbf{A(g(0,0))}^T&\mathbf{0}\\\mathbf{0}&\mathbf{I}_{15\times15}\end{bmatrix}\nonumber\\
    &=\begin{bmatrix}\mathbf{A}((\boldsymbol{\omega}_{m}-\mathbf{\widehat{b}}_{\boldsymbol{\omega}})\Delta{t})&\mathbf{0}\\\mathbf{0}&\mathbf{I}_{15\times15}\end{bmatrix}
    \label{eq:partialGpartialg}
\end{align}

Use the discrete dynamics in \eqref{eq:discrete_dynamics}, we have (here we omit the index $i$ for simplicity)
\begin{align}
    \mathbf{g}\left(\widetilde{\mathbf{x}},\mathbf{w}\right)
    &=\mathbf{f}\left(\mathbf{x},\mathbf{u},\mathbf{w}\right)\Delta t=\begin{bmatrix}\boldsymbol{\omega}_{m}-\mathbf{b}_{\boldsymbol{\omega}}-\mathbf{n}_{\boldsymbol{\omega}}\\{}^G{\mathbf{v}_{I}}\\{}^G\mathbf{R}_{I}\left(\mathbf{a}_{m}-\mathbf{b}_{\mathbf{a}}-\mathbf{n}_{\mathbf{a}}\right)+{}^{G}\mathbf{g}\\\mathbf{n}_{\mathbf{b}_{\boldsymbol{\omega}}}\\\mathbf{n}_{\mathbf{b}_{\mathbf{a}}}\\\mathbf{0}_{3\times1}\end{bmatrix}\Delta t\nonumber\\
    &=\begin{bmatrix}
        \boldsymbol{\omega}_{m}-\widehat{\mathbf{b}}_{\boldsymbol{\omega}}-\widetilde{\mathbf{b}}_{\boldsymbol{\omega}}-\mathbf{n}_{\boldsymbol{\omega}}\\
        {}^G{\widehat{\mathbf{v}}_{I}}+{}^G{\widetilde{\mathbf{v}}_{I}}\\
        {}^G\mathbf{\widehat{R}}_{I}\mathrm{Exp}(\delta\boldsymbol{\theta}^{T})\left(\mathbf{a}_{m}-\widehat{\mathbf{b}}_{\mathbf{a}}-\widetilde{\mathbf{b}}_{\mathbf{a}}-\mathbf{n}_{\mathrm{a}}\right)+{}^{G}\mathbf{g}\\
        \mathbf{n}_{\mathbf{b}_{\boldsymbol{\omega}}}\\\mathbf{n}_{\mathbf{b}_{\mathbf{a}}}\\
        \mathbf{0}_{3\times1}\end{bmatrix}\Delta t\label{eq:expression_g}
    \end{align}
Then
\[
\left.\frac{\partial \mathbf{g}(\widetilde{\mathbf{x}},\mathbf{0})}{\partial\widetilde{\mathbf{x}}}\right|_{\widetilde{\mathbf{x}}=\mathbf{0}}=\begin{pmatrix}\mathbf{0}&\mathbf{0}&\mathbf{0}&-\mathbf{I}_{3\times3}\Delta t&\mathbf{0}&\mathbf{0}\\\mathbf{0}&\mathbf{0}&\mathbf{I}_{3\times3}\Delta t&\mathbf{0}&\mathbf{0}&\mathbf{0}\\-{}^G\mathbf{\widehat{R}}_I\lfloor\mathbf{a}_m-\mathbf{\widehat{b}_a}\rfloor_\wedge\Delta t&\mathbf{0}&\mathbf{0}&\mathbf{0}&-{}^G\mathbf{\widehat{R}}_I\Delta t&\mathbf{I}_{3\times3}\Delta t\\\mathbf{0}&\mathbf{0}&\mathbf{0}&\mathbf{0}&\mathbf{0}&\mathbf{0}\\\mathbf{0}&\mathbf{0}&\mathbf{0}&\mathbf{0}&\mathbf{0}&\mathbf{0}\\\mathbf{0}&\mathbf{0}&\mathbf{0}&\mathbf{0}&\mathbf{0}&\mathbf{0}\end{pmatrix}
\]
Multiply the above equation with \eqref{eq:partialGpartialg} and add \eqref{eq:partialGpartialx}, we get $\mathbf{F}_{\mathbf{\widetilde{x}}}$.

From \eqref{eq:expression_g}, we also have
\[\left.\frac{\partial \mathbf{g}(\mathbf{0},\mathbf{w})}{\partial \mathbf{w}}\right|_{\mathbf{w}=\mathbf{0}}=\begin{pmatrix}-\mathbf{I}_{3\times3}\Delta t&\mathbf{0}&\mathbf{0}&\mathbf{0}\\\mathbf{0}&\mathbf{0}&\mathbf{0}&\mathbf{0}\\\mathbf{0}&-{}^G\mathbf{\widehat{R}}_I\Delta t&\mathbf{0}&\mathbf{0}\\\mathbf{0}&\mathbf{0}&\mathbf{I}_{3\times3}&\mathbf{0}\\\mathbf{0}&\mathbf{0}&\mathbf{0}&\mathbf{I}_{3\times3}\\\mathbf{0}&\mathbf{0}&\mathbf{0}&\mathbf{0}\end{pmatrix}\]

Then multiply the above equation with \eqref{eq:partialGpartialg}, we obtain the expression for $\mathbf{F}_{\mathbf{w}}$.

\printbibliography

\end{document}