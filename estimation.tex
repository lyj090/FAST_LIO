
\section{Estimation Algorithms}
\subsection{Unscented Kalman Filter (UKF)}
The UKF approximates the posterior distribution using sigma points. Key steps include:
\begin{algorithm}[H]
\caption{UKF Prediction and Update}
\begin{algorithmic}[1]
\State Compute sigma points $\chi_k$ from $\hat{\mathbf{x}}_{k|k}, \mathbf{P}_{k|k}$
\State Propagate through $f(\cdot)$: $\chi_{k+1|k} = f(\chi_k)$
\State Compute predicted mean $\hat{\mathbf{x}}_{k+1|k}$ and covariance $\mathbf{P}_{k+1|k}$
\State Generate measurement sigma points: $\gamma_k = h(\chi_{k+1|k})$
\State Compute predicted measurement $\hat{\mathbf{z}}_{k+1|k}$
\State Update state using Kalman gain $\mathbf{K}_{k+1}$
\end{algorithmic}
\end{algorithm}

\subsection{Extended Kalman Filter (EKF)}
For comparison, we implement the EKF using Jacobian linearization of $f(\cdot)$ and $h(\cdot)$.
