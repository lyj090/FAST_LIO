\section{Implementation and Experimental Setup}
\label{sec:implementation}

We successfully reproduced the core estimation pipeline of FAST-LIO~\cite{fast_lio}, including forward/backward propagation, motion undistortion, and the iterated Kalman update with the reformulated Kalman gain. 
The entire system is implemented in C++ for point cloud processing.

To ensure experimental reproducibility and environmental isolation, all tests were conducted within a Docker container running Ubuntu 20.04. The M3DGR dataset~\cite{m3dgr} is selected for its benchmark-quality LiDAR-IMU sequences and ground-truth poses from a high-precision RTK system.

We evaluate two key aspects:
\begin{enumerate}
    \item \textbf{Estimation accuracy}: Quantified by RMSE (Root Mean Square Error) after SE(3) alignment between the estimated trajectory and ground truth;
    \item \textbf{Computational efficiency}: Measured by per-scan processing time, comparing the conventional Kalman gain \eqref{eq:kalman_gain_standard} versus the efficient form derived from the Woodbury identity \eqref{eq:new_kalman_gain}.
\end{enumerate}

To validate the \textit{generality and robustness} of our implementation, we select two representative sequences from M3DGR:
\begin{itemize}
    \item \textbf{Bag 1 (indoor)}: 
    % TODO:Explain
    \item \textbf{Bag 2 (outdoor)}: 
    % TODO:Explain
\end{itemize}
This cross-scenario design ensures our conclusions are not biased toward a specific environment.

\section{Results and Analysis}
\label{sec:results}
% TODO:FIGURE REF

\subsection{Trajectory Accuracy and Error Statistics}
\label{subsec:accuracy}

We align the estimated trajectory to the ground truth using a global SE(3) transformation that minimizes the RMSE. The error statistics for Bag 1 are summarized in Table~\ref{tab:error_stats}.

\begin{table}[ht]
    \centering
    \begin{tabular}{lc}
    \toprule
    \textbf{Statistic} & \textbf{Value (m)} \\
    \midrule
    \textbf{RMSE} & \textbf{0.210} \\
    Mean & 0.202 \\
    Median & 0.195 \\
    Std. Dev ($\sigma$) & 0.059 \\
    Min & 0.050 \\
    Max & 0.364 \\
    SSE & 271.635 \\
    \bottomrule
    \end{tabular}
    \caption{Error statistics of the estimated trajectory on Bag 1 (indoor).}
    \label{tab:error_stats}
\end{table}

The low RMSE (0.210 m) and small standard deviation (0.059 m) indicate that our implementation achieves high-precision and stable estimation even in geometrically degenerate indoor environments. The maximum error (0.364 m) occurs during a rapid turn near a glass wall (where LiDAR features are scarce), confirming that feature degradation remains a fundamental challenge—though our tightly-coupled IEKF framework effectively contains the drift.

\noindent\textbf{p.s.} SSE denotes the Sum of Squared Errors, i.e., $\mathrm{SSE} = \sum_{k=1}^{N} \|\hat{\mathbf{p}}_k - \mathbf{p}_k^{\mathrm{gt}}\|^2$.

\subsection{Computational Efficiency of Kalman Gain Formulations}
\label{subsec:efficiency}

Figure~\ref{fig:time_comparison} and Table~\ref{tab:time_comparison} compare the processing latency per LiDAR scan using the two Kalman gain formulations.

\begin{figure}[ht]
    \centering
    \includegraphics[width=0.6\textwidth]{media/time_analize.pdf}
    \caption{Per-scan processing time comparison between the conventional (old) and Woodbury-optimized (new) Kalman gain formulas.}
    \label{fig:time_comparison}
\end{figure}

\begin{table}[ht]
    \centering
    \caption{Processing time (ms) per LiDAR scan.}
    \label{tab:time_comparison}
    \begin{tabular}{cccccc}
    \toprule
    \multirow{2}{*}{Method} & \multicolumn{2}{c}{Bag 1} & \multicolumn{2}{c}{Bag 2} \\
    \cmidrule(lr){2-3} \cmidrule(lr){4-5}
     & Max.(ms) & Avg.(ms) & Max.(ms) & Avg.(ms) \\
    \midrule
    Old Formula & 1633.65 & 305.35 & 221.78 & 57.48 \\
    New Formula & 2.97 & 0.40 & 2.39 & 0.33 \\
    \bottomrule
    \end{tabular}
\end{table}

The results are striking: the Woodbury-optimized formulation reduces average processing time by over \textbf{99.8\%}—from 305 ms to just 0.4 ms per scan in the indoor sequence. This dramatic improvement enables real-time operation at 50 Hz (LiDAR rate), whereas the conventional method fails to keep up, often dropping scans. In Bag 1, the old formula could only process the first $\sim$30\% of the trajectory before latency accumulated beyond the scan interval. In Bag 2 (sparser features, smaller $m$), it barely completed the sequence but with significant delay. This validates that the computational bottleneck of tight LiDAR-inertial fusion lies in the Kalman gain computation, and the proposed reformulation is essential for practical deployment.