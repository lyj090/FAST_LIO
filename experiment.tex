\section{Implementation and Experimental Setup}
\label{sec:implementation}
We successfully reproduced the FAST-LIO algorithm and conducted a comprehensive evaluation using the M3DGR dataset\cite{m3dgr}. 
To ensure reproducibility and isolation, all experiments were executed in a Docker environment running Ubuntu 20.04.

We mainly compared the performance by RMSE (Root Mean Square Error) and 
the run time whether with the Kalman gain in\ref{eq:new_kalman_gain} or the efficient form in\ref{eq:kalman_gain_standard}.

\section{Results and Analysis}
\label{sec:results}
效果展示

误差分析

运行时间分析

\begin{table}[htbp]
    \centering
    \begin{tabular}{cccccc}
    \toprule
    \multirow{2}{*}{Method} & \multicolumn{2}{c}{Bag 1} & \multicolumn{2}{c}{Bag 2} \\
    \cmidrule(lr){2-3} \cmidrule(lr){4-5}
     & Max.(ms) & Avg.(ms) & Max.(ms) & Avg.(ms) \\
    \midrule
    Old Formula & 1633.65 & 305.35 & 221.78 & 57.48 \\
    New Formula & 2.97 & 0.40 & 2.39 & 0.33 \\
    \bottomrule
    \end{tabular}
    \label{tab:time_comparison}
\end{table}

% TODO: figure
As shown in the figure, in both datasets, the absence of
the Woodbury identity significantly reduced FAST-
LIO’s processing speed. While the old formula failed
to complete Dataset 1 (allowing only about one-third of
the trajectory to be estimated), it managed to finish
Dataset 2—potentially because sparse point clouds
later eased the computational load.
